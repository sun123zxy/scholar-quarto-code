% --- lang test ---

\newif\ifzh
\RequirePackage{xstring}
\IfBeginWith{en}{zh}{
    \zhtrue
    \documentclass[fontset=fandol,envcountsect]{ctexbeamer}
    \def\transmodified{最后更新于}
    \def\transtoc{目录}
    \def\transreferences{参考文献}
}{
    \zhfalse
    \documentclass[fontset=fandol,envcountsect]{beamer}
    \def\transmodified{Last modified on}
    \def\transtoc{Table of Contents}
    \def\transreferences{References}
}

% --- pandoc ---

\usepackage{subcaption}


\providecommand{\tightlist}{%
  \setlength{\itemsep}{0pt}\setlength{\parskip}{0pt}}\usepackage{longtable,booktabs,array}
\usepackage{calc} % for calculating minipage widths
\usepackage{caption}
% Make caption package work with longtable
\makeatletter
\def\fnum@table{\tablename~\thetable}
\makeatother
\usepackage{graphicx}
\makeatletter
\newsavebox\pandoc@box
\newcommand*\pandocbounded[1]{% scales image to fit in text height/width
  \sbox\pandoc@box{#1}%
  \Gscale@div\@tempa{\textheight}{\dimexpr\ht\pandoc@box+\dp\pandoc@box\relax}%
  \Gscale@div\@tempb{\linewidth}{\wd\pandoc@box}%
  \ifdim\@tempb\p@<\@tempa\p@\let\@tempa\@tempb\fi% select the smaller of both
  \ifdim\@tempa\p@<\p@\scalebox{\@tempa}{\usebox\pandoc@box}%
  \else\usebox{\pandoc@box}%
  \fi%
}
% Set default figure placement to htbp
\def\fps@figure{htbp}
\makeatother

% --- adjust code font size ---


% --- bibliography ---

\RequirePackage[backend=biber,style=ieee,autocite=plain]{biblatex}
% \RequirePackage[backend=biber,style=gb7714-2015,autocite=plain]{biblatex}
\addbibresource{index.bib}

\makeatletter % force emph things in biblatex to be italic
\renewrobustcmd*{\mkbibemph}{\mkbibitalic}
\protected\long\def\blx@imc@mkbibemph#1{\blx@imc@mkbibitalic{#1}}
\makeatother

% -- tikz --

\usepackage{tikz}
\usepackage{tikz-cd}
\usepackage{quiver}

% --- theorems ---

\makeatletter
\def\cleartheorem#1{%
    \expandafter\let\csname#1\endcsname\relax
    \expandafter\let\csname c@#1\endcsname\relax
}
\makeatother

\cleartheorem{theorem}
\cleartheorem{lemma}
\cleartheorem{corollary}
\cleartheorem{definition}
\cleartheorem{example}
\cleartheorem{solution}

\ifzh % zhthm

\theoremstyle{theorem}
\newtheorem{theorem}{定理}
\newtheorem{definition}{定义}
\newtheorem{lemma}{引理}
\newtheorem{corollary}{推论}
\newtheorem{proposition}{命题}
\newtheorem{conjecture}{猜想}

\theoremstyle{example}
\newtheorem{example}{例}
\newtheorem{exercise}{习题}

\theoremstyle{remark}
\newtheorem*{myproof}{证明}
\newtheorem*{solution}{解}
\newtheorem*{remark}{注记}

\renewenvironment{proof}{\begin{myproof}}{\qed \end{myproof}}

\else % enthm

\theoremstyle{theorem}
\newtheorem{theorem}{Theorem}
\newtheorem{definition}{Definition}
\newtheorem{lemma}{Lemma}
\newtheorem{corollary}{Corollary}
\newtheorem{proposition}{Proposition}
\newtheorem{conjecture}{Conjecture}

\theoremstyle{example}
\newtheorem{example}{Example}
\newtheorem{exercise}{Exercise}

\theoremstyle{remark}
\newtheorem*{solution}{Solution}
\newtheorem*{remark}{Remark}

\fi % endthm

% \setbeamertemplate{theorems}[numbered]
% \setbeamertemplate{caption}[numbered]

% \addtobeamertemplate{proof begin}{\setbeamercolor{block title}{bg=black}}{}
\AtBeginEnvironment{proof}{\setbeamercolor{block title}{bg=black}}
\AtBeginEnvironment{solution}{\setbeamercolor{block title}{bg=black}}
\AtBeginEnvironment{remark}{\setbeamercolor{block title}{bg=black}}

% --- styles ---

\usetheme{Madrid}
\usefonttheme[onlymath]{serif}

\colorlet{beamer@blendedblue}{green!35!black}

\hypersetup{
    colorlinks=true,
    urlcolor=lime!80!black,
    linkcolor=, % default color for other links
    citecolor=lime!80!black,
    anchorcolor=,
    filecolor=
}

\newcommand{\myemph}[1]{{\usebeamercolor[fg]{structure} #1}}
\renewcommand{\emph}{\myemph}
\let\textbf\alert

\ifzh
    \setlength{\parindent}{2em}
    \addtobeamertemplate{block begin}{}{\setlength\parindent{2em}}
    \addtobeamertemplate{block example begin}{}{\setlength\parindent{2em}}
    \addtobeamertemplate{alertblock alerted begin}{}{\setlength\parindent{2em}}

    \setlength{\parskip}{3pt} % slighty more spacing
\fi

% --- structures ---

\AtBeginSection[]
{
  \begin{frame}{\transtoc}
    \tableofcontents[currentsection]
  \end{frame}
}
\AtBeginSubsection[]
{
  \begin{frame}{\transtoc}
    \tableofcontents[currentsubsection]
  \end{frame}
}

% --- main ---

\title{Classification of Quardratic Forms over \(\mathbb Q\)}
\subtitle{\textit{Algebraic Geometry}\thanks{A 2025 spring course lectured by Prof. Yangyu Fan.}
Final Presentation}
\author{Xingyu
Zhong\thanks{The speaker. Other group members are: Shubin Xue, Jiale Qiu, Binhong Sheng, Wenxuan Wu, Yuqing Wang}}
\institute[BIT]{Beijing Institute of Technology}
\date{2025-04-16}

\begin{document}

\frame{\titlepage}

\begin{frame}{\transtoc}
  \tableofcontents
\end{frame}

\section{First Attempts and General
Approachs}\label{first-attempts-and-general-approachs}

\begin{frame}{Foreword}
\phantomsection\label{foreword}
\newcommand{\T}{\mathrm{T}}
\newcommand{\Image}{\operatorname{Im}}
\newcommand{\rk}{\operatorname{rk}}
\newcommand{\sgn}{\operatorname{sgn}}
\newcommand{\rad}{\operatorname{rad}}

\begin{itemize}
\item
  References relied on heavily:

  \begin{itemize}
  \item
    J.P. Serre ``A Course in Arithmetic'' \autocite{serre_course_1973}
  \item
    Shiva Chidambaram, MIT18.782 Introduction to Arithmetic Geometry
    (Spring 2023)
    \href{https://math.mit.edu/classes/18.782/2023sp/LectureNotes9.pdf}{Lecture
    Notes}
  \item
    Arushi Gupta, Participant Papers of The University of Chicago
    Mathematics REU 2018,
    \href{https://math.uchicago.edu/~may/REU2018/REUPapers/Gupta.pdf}{The
    p-adic Integers, Analytically and Algebraically}
  \end{itemize}
\item
  May serve as a guidance of the first part of
  \autocite{serre_course_1973}
\item
  Assume familarity of quadratic residues and basic knowledge of
  \(p\)-adic numbers
\item
  Skip most of the proofs
\item
  Apology in advance for potential mistakes
\end{itemize}
\end{frame}

\begin{frame}{Notations}
\phantomsection\label{notations}
\begin{itemize}
\item
  We denote by \(K\) an arbitrary field. \textbf{All fields are assumed
  to be of characteristic \(\neq 2\)}.
\item
  \(\nu_p : \mathbb Q_p \to \mathbb Z\) being the \(p\)-adic valuation.
\item
  \(\left(\dfrac a p \right)\) being the Legendre symbol. \(a\) is
  understanded as \(p^{-\nu_p(a)} a \bmod p\) if \(a \in \mathbb Q_p\).
  Define this similarly in \(\mathbb F_q\).
\item
  Let \(f \oplus g\) denote the direct sum of two quadratic forms \(f\)
  and \(g\).
\end{itemize}
\end{frame}

\subsection{\texorpdfstring{Example: Quadratic Forms over \(\mathbb R\)
and
\(\mathbb F_q\)}{Example: Quadratic Forms over \textbackslash mathbb R and \textbackslash mathbb F\_q}}\label{example-quadratic-forms-over-mathbb-r-and-mathbb-f_q}

\begin{frame}{Review: Quadratic Forms over \(\mathbb R\)}
\phantomsection\label{review-quadratic-forms-over-mathbb-r}
\begin{itemize}
\item
  A quadratic form \(f: V \to K\) may be identified by a symmetric
  matrix \(A \in M_n(K)\) by \(f(v) = v^\mathrm{T}A v\).

  Their equivalence is defined by \emph{congruence}:
  \(A \sim B \iff A = Q^\mathrm{T}B Q\).
\item
  Real symmetric matrices may be diagonalized orthogonally.
\item
  Scale each eigenvalue by multiplying a square. Only their sign
  matters.

  \begin{itemize}
  \tightlist
  \item
    the \emph{rank} \(n\), an invariant
  \item
    the \emph{signature}
    \((r,s) := (\# \text{positive eigenvalues}, \# \text{negative eigenvalues})\).
  \end{itemize}
\item
  Same rank and signature implies the equivalence.
\item
  Sylvester's law of inertia: signature is also an invariant.
\end{itemize}
\end{frame}

\begin{frame}{Some Refinement}
\phantomsection\label{some-refinement}
On an abitrary field \(K\):

\begin{itemize}
\item
  All symmetric matrix is equivalent to a diagonal one.

  \begin{itemize}
  \tightlist
  \item
    Pick a non-isotropic vector \(v\) (exists when the form is nonzero),
    its orthogonal complement is a hyperplane and does not include
    \(v\). Change basis and do the induction.
  \end{itemize}
\item
  The rank is always a invariant. \textbf{We may (and we shall always)
  reduce to classify the non-degenerate quadratic forms of rank \(n\).}
\item
  The squares \((K^\times)^2\) give us the ability to scale. Knowledge
  of distribution of diagonal elements in \(K^\times / (K^\times)^2\)
  suffices to show the equivalence\footnote<.->[frame]{Though working in
    the refined structure \(\{0\} \cup K^\times / (K^\times)^2\) is
    probably a better idea if one wish to deal with the degenerate case
    in a uniform manner.}.

  \begin{itemize}
  \tightlist
  \item
    \(\mathbb C^\times / (\mathbb C^\times)^2 \cong \{1\}\), suffices to
    classify by the rank.
  \item
    \(\mathbb R^\times / (\mathbb R^\times)^2 \cong \{1,-1\}\),
    signature is also needed.
  \item
    \(\mathbb F_q^\times / (\mathbb F_q^\times)^2 \cong \{1,a\}\), where
    \(a \in \mathbb F_q\) is a quadratic nonresidue.
  \item
    For \(\mathbb Q_p\) and \(\mathbb Q\)?
  \end{itemize}
\end{itemize}
\end{frame}

\begin{frame}{Another Example\footnote<.->[frame]{We would like to thank
  Prof.~Yu Zhao for his advice on this part.}: Quadratic Forms over
\(\mathbb F_q\)}
\phantomsection\label{another-example-quadratic-forms-over-mathbb-f_q}
We classify the non-degenerate quadratic forms of rank \(n\).

\begin{itemize}
\item
  Refined signature: counting nonzero quadratic residues and
  nonresidues. It may serve as a sufficient criterion of the
  equivalence.
\item
  But it's not an invariant. \(a X^2 + a Y^2 \sim X^2 + Y^2\) over
  \(\mathbb F_q\).

  \begin{itemize}
  \item
    Do a change of basis \(X=sU+tV\) and \(Y=tU-sV\). If we require
    \(a U^2 + a V^2 = X^2 + Y^2\), then \(s^2+t^2=a\).
  \item
    It always has a nonzero solution in \(\mathbb F_q\): \(s^2\) and
    \(a - t^2\) have both \((q+1)/2\) possible values, thus must reach a
    common value.
  \end{itemize}
\item
  The \emph{discriminant}
  \(d := \left( \dfrac{\det(A)}{q} \right) \in \mathbb F_q^\times / (\mathbb F_q^\times)^2\)
  is an invariant and reveals the parity of the signature. It classifies
  the non-degenerate quadratic forms over \(\mathbb F_q\).
\end{itemize}

Insight: Existence of nonzero solutions of the equation
\(a X^2 + b Y^2 = Z^2\) in \(K\) seems to be of great importantance.
\end{frame}

\subsection{Quadratic Spaces}\label{quadratic-spaces}

\begin{frame}{Quadratic Spaces}
\phantomsection\label{quadratic-spaces-1}
The structure of a quadratic space, i.e.~vector space equipped with a
symmetric bilinear form, is much more subtle than its positive-definite
counterpart over \(\mathbb R\) or \(\mathbb C\). For example, for a
non-degenerate quadratic space \(V\) and a subspace \(U\) of \(V\)
(\autocite{serre_course_1973} p.~28, chap.~4, sec.~1.2):

\begin{itemize}
\item
  \(U \cap U^\perp = \operatorname{rad}(U)\),
  \(\dim U + \dim U^\perp = \dim V\), \((U^\perp)^\perp = U\)
\item
  \(U \oplus U^\perp = V\) iff \(U + U^\perp = V\) iff
  \(\operatorname{rad}(U) = 0\)
\item
  It's much harder to show that an orthogonal basis of \(U\) expands to
  an orthogonal basis of \(V\).
\end{itemize}
\end{frame}

\begin{frame}{Structure of Quadratic Spaces}
\phantomsection\label{structure-of-quadratic-spaces}
We mention some results here without details.

\begin{theorem}[Witt (\autocite{serre_course_1973} p.~31, chap.~4,
sec.~1.5, theorem 3)]\protect\hypertarget{thm-witt}{}\label{thm-witt}

Every injective metric-preserving map from a subspace \(U\) of a
quadratic space \(V\) to another quadratic space \(W\) may be extended
to a metric-preserving map from \(V\) to \(W\).

\end{theorem}

\begin{theorem}[Witt's cancellation (\autocite{serre_course_1973} p.~34,
chap.~4, sec.~1.6, theorem
4)]\protect\hypertarget{thm-witt-cancel}{}\label{thm-witt-cancel}

\(f_1 \oplus g_1 \sim f_2 \oplus g_2\) and \(g_1 \sim g_2\) implies
\(f_1 \sim f_2\).

\end{theorem}

\begin{theorem}[Witt's
decomposition]\protect\hypertarget{thm-witt-decomposition}{}\label{thm-witt-decomposition}

Every quadratic space \(V\) is a direct sum of:
\(\operatorname{rad}(V)\), an anisotropic quadratic space (i.e.~its
nonzero vectors has nonzero norms) and a split quadratic space
(i.e.~\(U = U^\perp\), full of hyperbolas)

\end{theorem}
\end{frame}

\subsection{The Common Represented Element
Method}\label{the-common-represented-element-method}

\begin{frame}{Another Invariant: The Range}
\phantomsection\label{another-invariant-the-range}
On an abitrary field \(K\), we say that a quadratic form \(f\)
\emph{represent}s \(a \in K\) if there exists a nonzero \(v \in V\) such
that \(f(v) = a\).

\begin{itemize}
\item
  The range of \(f\), \(\operatorname{Im}f\), is an invariant.
\item
  It may be viewed in \(\{0\} \cup K^\times / (K^{\times})^2\).
\item
  Is it complete?
\end{itemize}
\end{frame}

\begin{frame}{Insights from the Range}
\phantomsection\label{insights-from-the-range}
\begin{proposition}[(\autocite{serre_course_1973} p.~33, chap.~4,
sec.~1.6, corollary
1)]\protect\hypertarget{prp-range}{}\label{prp-range}

Let \(a \in K^\times\). TFAE:

\begin{itemize}
\tightlist
\item
  \(f\) represents \(a\)
\item
  \(f \sim g \oplus (Z \mapsto a Z^2)\) where \(g\) is of rank
  \(\operatorname{rk}f - 1\).
\item
  \(f \oplus (Z \mapsto -a Z^2)\) represents \(0\).
\end{itemize}

\end{proposition}

\begin{itemize}
\tightlist
\item
  Insight from line 3: To understand the range, it suffices to examine
  when a quadratic form represents \(0\).
\item
  Insight from line 2 (\emph{the common represented element method}):
  Say \(f_1\), \(f_2\) are nozero and represent a common
  \(a \in K^\times\). Reducing \(Z \mapsto aZ^2\), if only \(g_1\) and
  \(g_2\) also share a common represented element\ldots{}
\end{itemize}
\end{frame}

\begin{frame}{Insights from the Range}
\phantomsection\label{insights-from-the-range-1}
\begin{itemize}
\item
  Sadly, range is not always a complete invariant.

  \begin{itemize}
  \tightlist
  \item
    Otherwise all indefinite quadratic forms over \(\mathbb R\) are
    equivalent, absurd.
  \end{itemize}
\item
  But we shall show that when \(K = \mathbb Q_p\) and moreover
  \(K = \mathbb Q\), it plays a subtle role in the classification of
  quadratic forms. This requires a more precise characterization of the
  range.
\item
  In fact, if only there are some simple invariants that can fully
  determines the range\ldots{}
\end{itemize}
\end{frame}

\subsection{Global and Local
Equivalence}\label{global-and-local-equivalence}

\begin{frame}{Global and Local Equivalence}
\phantomsection\label{global-and-local-equivalence-1}
\begin{itemize}
\item
  Fact: Field extensions preserve the equivalence of quadratic forms.

  \begin{itemize}
  \tightlist
  \item
    Example: Equivalence classes are finer over \(\mathbb R\) than those
    over \(\mathbb C\).
  \end{itemize}
\item
  \(\mathbb Q \hookrightarrow \mathbb R\), thus the rank and the
  signature are invariants. But we need more information to classify.
\item
  Other field extension of \(\mathbb Q\)?
  \(\mathbb Q \hookrightarrow \mathbb Q_p\)
\end{itemize}

\begin{theorem}[Hasse-Minkowski (\autocite{serre_course_1973} p.~41,
chap.~4, sec.~3.1, theorem
8)]\protect\hypertarget{thm-hasse-minkowski}{}\label{thm-hasse-minkowski}

\(f\) represents \(0\) over \(\mathbb Q\) iff it represents \(0\) over
\(\mathbb R\) and all \(\mathbb Q_p\).

\end{theorem}

\begin{itemize}
\tightlist
\item
  To gain more invariants for \(\mathbb Q\) (especially those related to
  the range), let's classify quadratic forms over \(\mathbb Q_p\) first.
\end{itemize}
\end{frame}

\section{\texorpdfstring{Quadratic Forms over \(\mathbb Q_p\) and
\(\mathbb Q\)}{Quadratic Forms over \textbackslash mathbb Q\_p and \textbackslash mathbb Q}}\label{quadratic-forms-over-mathbb-q_p-and-mathbb-q}

\subsection{\texorpdfstring{Structure of
\(\mathbb Q_p^\times / (\mathbb Q_p^\times)^2\)}{Structure of \textbackslash mathbb Q\_p\^{}\textbackslash times / (\textbackslash mathbb Q\_p\^{}\textbackslash times)\^{}2}}\label{structure-of-mathbb-q_ptimes-mathbb-q_ptimes2}

\begin{frame}{Structure of \(\mathbb Q_p^\times\)}
\phantomsection\label{structure-of-mathbb-q_ptimes}
\begin{itemize}
\item
  \(\mathbb Q_p^\times \cong \mathbb Z \times \mathbb Z_p^\times\) by
  collecting common powers of \(p\)
\item
  \(\mathbb Z_p^\times \cong \mathbb F_p^\times \times (1 + p \mathbb Z_p)\)
  by \(a \mapsto a \bmod p\)

  \begin{itemize}
  \tightlist
  \item
    It splits by the explicit construction of a primitive root of order
    \(p\), via Hensel's lemma / Teichmüller lift
    \(\lim_{n \to \infty} g^{p^n}\), where \(g\) is a primitive root of
    \(\mathbb F_p^\times\).
  \end{itemize}
\end{itemize}
\end{frame}

\begin{frame}{Structure of \(1 + p \mathbb Z_p\) and the log / exp map}
\phantomsection\label{structure-of-1-p-mathbb-z_p-and-the-log-exp-map}
For \(p \neq 2,\, \alpha \geq 1\) or \(p = 2,\, \alpha \geq 2\): \[
\begin{aligned}
1 + p^\alpha \mathbb Z_p &\cong (p^\alpha \mathbb Z_p, +) \cong (\mathbb Z_p, +) \\
1 + p^\alpha a &\mapsto \log(1+p^\alpha a)
\end{aligned}
\]

For \(p=2,\,\alpha = 1\),

\begin{itemize}
\item
  \[
  1 + 2 \mathbb Z_2 \cong \mathbb Z / 2 \mathbb Z \times (1 + 4 \mathbb Z_2)
  \]

  \begin{itemize}
  \item
    by \(1 + 2a \mapsto a \bmod 2\)
  \item
    It splits by the explicit construction of a primitive root of order
    \(2\): \((-1,-1,\dots) = \sum_{n=0}^{+\infty} 2^n\).
  \end{itemize}
\item
  \((1 + 4 \mathbb Z_2) \cong (4 \mathbb Z_2, +) \cong (\mathbb Z_2, +)\)
  by the log map
\item
  Thus \[
  1 + 2 \mathbb Z_2 \cong \mathbb Z / 2 \mathbb Z \times (\mathbb Z_2, +)
  \]
\end{itemize}
\end{frame}

\begin{frame}{Quadratic residues of \(\mathbb Q_p\)}
\phantomsection\label{quadratic-residues-of-mathbb-q_p}
For \(p \neq 2\):

\begin{itemize}
\item
  \(\mathbb Q_p^\times \cong \mathbb Z \times \mathbb F_p^\times \times (\mathbb Z_p, +)\)
\item
  \(2\) is a unit in \(\mathbb Z_p\). Thus
  \(a \in (\mathbb Q_p^\times)^2\) iff \(\nu_p(a) \bmod 2 = 0\) and
  \(a \bmod p \in \mathbb F_p^\times\) is a quadratic residue.
\item
  \(\mathbb Q_p^\times / (\mathbb Q_p^\times)^2 \cong \mathbb Z / 2\mathbb Z \times \mathbb Z / 2\mathbb Z\),
  generated by \(p\) and \(a\), where \(a \bmod p\) is a quadratic
  nonresidue.
\end{itemize}

For \(p = 2\):

\begin{itemize}
\item
  \(\mathbb Q_2^\times \cong \mathbb Z \times \mathbb Z / 2 \mathbb Z \times (\mathbb Z_2, +)\)
\item
  Quadratic residues of \((\mathbb Z_2, +)\) are \((2 \mathbb Z_2, +)\),
  which pulls back to \(1 + 8 \mathbb Z_2\).
\item
  \(a \in (\mathbb Q_2^\times)^2\) iff \(\nu_2(a) \bmod 2 = 0\) and
  \(a \bmod 8 \equiv 1\).
\item
  \(\mathbb Q_2^\times / (\mathbb Q_2^\times)^2 \cong \mathbb Z / 2\mathbb Z \times \mathbb Z / 2\mathbb Z \times \mathbb Z / 2\mathbb Z\),
  generated by \(2\), \(3\) and \(5\).
\end{itemize}
\end{frame}

\subsection{The Hilbert Symbol}\label{the-hilbert-symbol}

\begin{frame}{The Hilbert Symbol}
\phantomsection\label{the-hilbert-symbol-1}
The Hilbert symbol over \(\mathbb Q_p\) is defined as: \[
\langle a, b \rangle := \begin{cases}
  1 & \text{if } a X^2 + b Y^2 = Z^2 \text{ has a nonzero solution in } \mathbb Q_p \\
  -1 & \text{otherwise}
\end{cases}
\] The symbol may be also viewed in
\(\{0\} \cup \mathbb Q_p^\times / (\mathbb Q_p^\times)^2\) or even more
simply in \(\mathbb Q_p^\times / (\mathbb Q_p^\times)^2\) when working
with non-degenerate forms.\footnote<.->[frame]{Lots of the resources,
  even \autocite{serre_course_1973}, switch between these three views
  without enough warning. Sadly we shall also commit this ususal mild
  sin (and have already done to other innocent invariants such as the
  discriminant\ldots)}
\end{frame}

\begin{frame}{Properties of the Hilbert Symbol}
\phantomsection\label{properties-of-the-hilbert-symbol}
\begin{itemize}
\item
  \(\langle a, -a \rangle = 1\)
\item
  \(\langle a, b \rangle = \langle b, a \rangle\) (symmetric)
\item
  If \(\langle a_2,b \rangle = 1\), then
  \(\langle a_1 a_2, b \rangle = \langle a_1,b \rangle\)

  \begin{itemize}
  \tightlist
  \item
    In fact,
    \(\langle a_1 a_2, b \rangle = \langle a_1, b \rangle \langle a_2, b \rangle\)
    (multiplicatively bilinear)
  \end{itemize}
\item
  \(\langle a, b \rangle = 1\) for all \(b\) iff \(a \in \mathbb Q_p^2\)
  (nondegenerate)
\item
  the Hilbert symbol is a non-degenerate symmetric bilinear form of the
  \(\mathbb F_2\)-vector space
  \(\mathbb Q_p^\times / (\mathbb Q_p^\times)^2\)

  \begin{itemize}
  \item
    This is a non-trivial result and is said to be, to some extent, a
    generalization of the law of quadratic reciprocity in local class
    field theory.
  \item
    To show above over \(\mathbb Q_p\), we develop an explicit formula
    for the Hilbert symbol.
  \end{itemize}
\end{itemize}
\end{frame}

\begin{frame}{The Explicit Formula of the Hilbert Symbol}
\phantomsection\label{the-explicit-formula-of-the-hilbert-symbol}
\begin{theorem}[(\autocite{serre_course_1973} p.~20, chap.~3, sec.~1.2,
theorem
1)]\protect\hypertarget{thm-hilbert-symbol}{}\label{thm-hilbert-symbol}

Say \(a = p^\alpha u\) and \(b = p^\beta v\) are \(p\)-adic numbers
where \(u,v \in \mathbb Z_p^\times\), then \[
\langle a, b \rangle = (-1)^{\alpha \cdot \beta \cdot \frac{p-1}{2}} \left(\dfrac{u}{p}\right)^\beta \left(\dfrac{v}{p}\right)^\alpha \text{ if } p \neq 2
\] We omit the case \(p=2\). It's a tedious modification of the above
formula.

\end{theorem}

\begin{longtable}[]{@{}lcccc@{}}
\toprule\noalign{}
& \(0\) & \(1\) & \(a\) & \(p\) \\
\midrule\noalign{}
\endfirsthead
\toprule\noalign{}
& \(0\) & \(1\) & \(a\) & \(p\) \\
\midrule\noalign{}
\endhead
\(0\) & \(1\) & \(1\) & \(1\) & \(1\) \\
\(1\) & & \(1\) & \(1\) & \(1\) \\
\(a\) & & & \(1\) & \(-1\) \\
\(p\) & & & & \((-1)^{\frac{p-1}{2}}\) \\
\bottomrule\noalign{}
\caption{Table of Hilbert symbol, \(p \neq 2\)}\tabularnewline
\end{longtable}
\end{frame}

\subsection{Invariants that Determines the
Range}\label{invariants-that-determines-the-range}

\begin{frame}{The Hasse Invariant}
\phantomsection\label{the-hasse-invariant}
Recall that we have reduced to work with non-degenerate diagonalized
quadratic forms of rank \(n\). Recall that the discriminant \[
d(f) = a_1 a_2 \dots a_n \in \mathbb Q_p^\times / (\mathbb Q_p^\times)^2
\] is an invariant.

\begin{itemize}
\item
  Define the \emph{Hasse invariant}
  \(\varepsilon(f) := \prod_{1 \leq i < j \leq n} \langle a_i, a_j \rangle\)
\item
  It is an invariant: \[
  \varepsilon(f) = \prod_{1 \leq i < j \leq n} \langle a_i, a_j \rangle = \varepsilon(f_1) \prod_{2 \leq j \leq n} \langle a_1, a_j \rangle = \varepsilon(f_1) \cdot \langle a_1, a_1 d(f) \rangle
  \] Thus \(\varepsilon\) is preserved under \emph{contiguous} change of
  orthogonal bases (fixes one of the vector of the basis)

  \begin{itemize}
  \tightlist
  \item
    For \(n \geq 3\), orthorgonal bases are transitive under contiguous
    change (\autocite{serre_course_1973} p.~30, sec.~4.1.4, theorem 2)
  \end{itemize}
\end{itemize}
\end{frame}

\begin{frame}{\(d\) and \(\varepsilon\) Determine the Range}
\phantomsection\label{d-and-varepsilon-determine-the-range}
\begin{theorem}[(\autocite{serre_course_1973} p.~36, chap.~4, sec.~2.2,
theorem
6)]\protect\hypertarget{thm-qp-range-determine}{}\label{thm-qp-range-determine}

For a non-degenerate quadratic form \(f\) of rank \(n\) over
\(\mathbb Q_p\), the range of \(f\) is determined by the discriminant
\(d := d(f)\) and the Hasse invariant \(\varepsilon := \varepsilon(f)\).

Or, in detail, \(f\) represents \(0\) iff:

\begin{itemize}
\tightlist
\item
  For \(n=2\): \(d=-1\)
\item
  For \(n=3\): \(\langle -1,-d \rangle = \varepsilon\)
\item
  For \(n=4\): \(d \neq 1\) or \(d=1\) and
  \(\varepsilon = \langle -1,-1 \rangle\)
\item
  For \(n=5\): no conditions
\end{itemize}

\end{theorem}

Recall that \(f\) represents
\(a \in \mathbb Q_p^\times / (\mathbb Q_p^\times)^2\) iff
\(f \oplus (Z \mapsto -a Z^2)\) represents \(0\), thus above fully
characterizes the range.
\end{frame}

\subsection{Classification}\label{classification}

\begin{frame}{Classification of Quadratic Forms over \(\mathbb Q_p\)}
\phantomsection\label{classification-of-quadratic-forms-over-mathbb-q_p}
\begin{theorem}[(\autocite{serre_course_1973} p.~39, chap.~4, sec.~2.3,
theorem
7)]\protect\hypertarget{thm-qp-classification}{}\label{thm-qp-classification}

Two non-degenerate quadratic forms of rank \(n\) over \(\mathbb Q_p\)
are equivalent iff they have the same discriminant \(d\) and Hasse
invariant \(\varepsilon\).

\end{theorem}

\begin{itemize}
\item
  \(f,g\) have same \(d\) and \(\varepsilon\), thus have the same range.
  Say they both represent
  \(a \in \mathbb Q_p^\times / (\mathbb Q_p^\times)^2\).
\item
  Then \(f \sim f_1 \oplus (Z \mapsto a Z^2)\), where \(f_1\) is of rank
  \(n-1\).
\item
  \(d\) and \(\varepsilon\) of \(f_1\) can be determined:

  \begin{itemize}
  \tightlist
  \item
    \(d(f_1) = a d(f)\)
  \item
    \(\varepsilon(f_1) = \varepsilon(f) \cdot (a, a d(f))\) (shown when
    discussing the invariance of \(\varepsilon\))
  \end{itemize}
\item
  The same for \(g\). Thus \(f_1, g_1\) shares the same \(d\) and
  \(\varepsilon\) (thus also their range). QED by induction.
\end{itemize}
\end{frame}

\begin{frame}{Classification of Quadratic Forms over \(\mathbb Q\)}
\phantomsection\label{classification-of-quadratic-forms-over-mathbb-q}
\begin{theorem}[(\autocite{serre_course_1973} p.~39, chap.~4, sec.~2.3,
theorem
7)]\protect\hypertarget{thm-qp-classification}{}\label{thm-qp-classification}

Two non-degenerate quadratic forms of rank \(n\) over \(\mathbb Q_p\)
are equivalent iff they are equivalent over \(\mathbb R\) and over each
\(\mathbb Q_p\).

\end{theorem}

\begin{itemize}
\item
  Say \(f,g\) are equivalent over each local field (\(\mathbb Q_p\) and
  \(\mathbb R\)), thus they share the same range locally.
\item
  By Hasse-Minkowski theorem, they also share the same range globally
  over \(\mathbb Q\).
\item
  Then \(f \sim f_1 \oplus (Z \mapsto a Z^2)\) globally, where \(f_1\)
  is of rank \(n-1\). The same for \(g\).
\item
  \(f_1 \sim g_1\) locally by Witt's cancellation theorem. QED by
  induction.
\end{itemize}
\end{frame}

\begin{frame}{Problem Remains}
\phantomsection\label{problem-remains}
\begin{itemize}
\item
  Proof of the Hasse-Minkowski theorem

  \begin{itemize}
  \tightlist
  \item
    essentially needs some understanding of the global property of the
    Hilbert symbol, which we have not discussed (cf.
    \autocite{serre_course_1973})
  \end{itemize}
\item
  Refine the theory for degenerate quadratic forms (relatively easy)
\item
  Enumerate all the equivalence classes of quadratic forms over
  \(\mathbb Q_p\) and \(\mathbb Q\) (cf. \autocite{serre_course_1973})
\item
  To what extent can we use the common represented element method to
  classify quadratic forms over other fields?
\item
  For which fields, the range of a quadratic form is a complete
  invariant? (At least \(\mathbb R\) fails. \(\mathbb Q\),
  \(\mathbb Q_p\)?)
\item
  What can we say about \(\mathbb Q^\times / (\mathbb Q^\times)^2\)?
\item
  Classification of quadratic forms over commutative rings
  (e.g.~\(\mathbb Z\), \(\mathbb Z / m \mathbb Z\))
\end{itemize}
\end{frame}

\begin{frame}[allowframebreaks]{\transreferences}
    \printbibliography[heading=none]
\end{frame}

\end{document}
