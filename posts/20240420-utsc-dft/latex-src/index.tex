% Options for packages loaded elsewhere
\PassOptionsToPackage{unicode}{hyperref}
\PassOptionsToPackage{hyphens}{url}
%
\documentclass[
  ignorenonframetext,
  chinese-hans,
]{beamer}
\usepackage{pgfpages}
\setbeamertemplate{caption}[numbered]
\setbeamertemplate{caption label separator}{: }
\setbeamercolor{caption name}{fg=normal text.fg}
\beamertemplatenavigationsymbolsempty
% Prevent slide breaks in the middle of a paragraph
\widowpenalties 1 10000
\raggedbottom
\setbeamertemplate{part page}{
  \centering
  \begin{beamercolorbox}[sep=16pt,center]{part title}
    \usebeamerfont{part title}\insertpart\par
  \end{beamercolorbox}
}
\setbeamertemplate{section page}{
  \centering
  \begin{beamercolorbox}[sep=12pt,center]{section title}
    \usebeamerfont{section title}\insertsection\par
  \end{beamercolorbox}
}
\setbeamertemplate{subsection page}{
  \centering
  \begin{beamercolorbox}[sep=8pt,center]{subsection title}
    \usebeamerfont{subsection title}\insertsubsection\par
  \end{beamercolorbox}
}
\AtBeginPart{
  \frame{\partpage}
}
\AtBeginSection{
  \ifbibliography
  \else
    \frame{\sectionpage}
  \fi
}
\AtBeginSubsection{
  \frame{\subsectionpage}
}

\usepackage{amsmath,amssymb}
\usepackage{iftex}
\ifPDFTeX
  \usepackage[T1]{fontenc}
  \usepackage[utf8]{inputenc}
  \usepackage{textcomp} % provide euro and other symbols
\else % if luatex or xetex
  \usepackage{unicode-math}
  \defaultfontfeatures{Scale=MatchLowercase}
  \defaultfontfeatures[\rmfamily]{Ligatures=TeX,Scale=1}
\fi
\usepackage{lmodern}
\ifPDFTeX\else  
    % xetex/luatex font selection
\fi
% Use upquote if available, for straight quotes in verbatim environments
\IfFileExists{upquote.sty}{\usepackage{upquote}}{}
\IfFileExists{microtype.sty}{% use microtype if available
  \usepackage[]{microtype}
  \UseMicrotypeSet[protrusion]{basicmath} % disable protrusion for tt fonts
}{}
\makeatletter
\@ifundefined{KOMAClassName}{% if non-KOMA class
  \IfFileExists{parskip.sty}{%
    \usepackage{parskip}
  }{% else
    \setlength{\parindent}{0pt}
    \setlength{\parskip}{6pt plus 2pt minus 1pt}}
}{% if KOMA class
  \KOMAoptions{parskip=half}}
\makeatother
\usepackage{xcolor}
\newif\ifbibliography
\setlength{\emergencystretch}{3em} % prevent overfull lines
\setcounter{secnumdepth}{2}

\usepackage{color}
\usepackage{fancyvrb}
\newcommand{\VerbBar}{|}
\newcommand{\VERB}{\Verb[commandchars=\\\{\}]}
\DefineVerbatimEnvironment{Highlighting}{Verbatim}{commandchars=\\\{\}}
% Add ',fontsize=\small' for more characters per line
\usepackage{framed}
\definecolor{shadecolor}{RGB}{241,243,245}
\newenvironment{Shaded}{\begin{snugshade}}{\end{snugshade}}
\newcommand{\AlertTok}[1]{\textcolor[rgb]{0.68,0.00,0.00}{#1}}
\newcommand{\AnnotationTok}[1]{\textcolor[rgb]{0.37,0.37,0.37}{#1}}
\newcommand{\AttributeTok}[1]{\textcolor[rgb]{0.40,0.45,0.13}{#1}}
\newcommand{\BaseNTok}[1]{\textcolor[rgb]{0.68,0.00,0.00}{#1}}
\newcommand{\BuiltInTok}[1]{\textcolor[rgb]{0.00,0.23,0.31}{#1}}
\newcommand{\CharTok}[1]{\textcolor[rgb]{0.13,0.47,0.30}{#1}}
\newcommand{\CommentTok}[1]{\textcolor[rgb]{0.37,0.37,0.37}{#1}}
\newcommand{\CommentVarTok}[1]{\textcolor[rgb]{0.37,0.37,0.37}{\textit{#1}}}
\newcommand{\ConstantTok}[1]{\textcolor[rgb]{0.56,0.35,0.01}{#1}}
\newcommand{\ControlFlowTok}[1]{\textcolor[rgb]{0.00,0.23,0.31}{\textbf{#1}}}
\newcommand{\DataTypeTok}[1]{\textcolor[rgb]{0.68,0.00,0.00}{#1}}
\newcommand{\DecValTok}[1]{\textcolor[rgb]{0.68,0.00,0.00}{#1}}
\newcommand{\DocumentationTok}[1]{\textcolor[rgb]{0.37,0.37,0.37}{\textit{#1}}}
\newcommand{\ErrorTok}[1]{\textcolor[rgb]{0.68,0.00,0.00}{#1}}
\newcommand{\ExtensionTok}[1]{\textcolor[rgb]{0.00,0.23,0.31}{#1}}
\newcommand{\FloatTok}[1]{\textcolor[rgb]{0.68,0.00,0.00}{#1}}
\newcommand{\FunctionTok}[1]{\textcolor[rgb]{0.28,0.35,0.67}{#1}}
\newcommand{\ImportTok}[1]{\textcolor[rgb]{0.00,0.46,0.62}{#1}}
\newcommand{\InformationTok}[1]{\textcolor[rgb]{0.37,0.37,0.37}{#1}}
\newcommand{\KeywordTok}[1]{\textcolor[rgb]{0.00,0.23,0.31}{\textbf{#1}}}
\newcommand{\NormalTok}[1]{\textcolor[rgb]{0.00,0.23,0.31}{#1}}
\newcommand{\OperatorTok}[1]{\textcolor[rgb]{0.37,0.37,0.37}{#1}}
\newcommand{\OtherTok}[1]{\textcolor[rgb]{0.00,0.23,0.31}{#1}}
\newcommand{\PreprocessorTok}[1]{\textcolor[rgb]{0.68,0.00,0.00}{#1}}
\newcommand{\RegionMarkerTok}[1]{\textcolor[rgb]{0.00,0.23,0.31}{#1}}
\newcommand{\SpecialCharTok}[1]{\textcolor[rgb]{0.37,0.37,0.37}{#1}}
\newcommand{\SpecialStringTok}[1]{\textcolor[rgb]{0.13,0.47,0.30}{#1}}
\newcommand{\StringTok}[1]{\textcolor[rgb]{0.13,0.47,0.30}{#1}}
\newcommand{\VariableTok}[1]{\textcolor[rgb]{0.07,0.07,0.07}{#1}}
\newcommand{\VerbatimStringTok}[1]{\textcolor[rgb]{0.13,0.47,0.30}{#1}}
\newcommand{\WarningTok}[1]{\textcolor[rgb]{0.37,0.37,0.37}{\textit{#1}}}

\providecommand{\tightlist}{%
  \setlength{\itemsep}{0pt}\setlength{\parskip}{0pt}}\usepackage{longtable,booktabs,array}
\usepackage{calc} % for calculating minipage widths
\usepackage{caption}
% Make caption package work with longtable
\makeatletter
\def\fnum@table{\tablename~\thetable}
\makeatother
\usepackage{graphicx}
\makeatletter
\newsavebox\pandoc@box
\newcommand*\pandocbounded[1]{% scales image to fit in text height/width
  \sbox\pandoc@box{#1}%
  \Gscale@div\@tempa{\textheight}{\dimexpr\ht\pandoc@box+\dp\pandoc@box\relax}%
  \Gscale@div\@tempb{\linewidth}{\wd\pandoc@box}%
  \ifdim\@tempb\p@<\@tempa\p@\let\@tempa\@tempb\fi% select the smaller of both
  \ifdim\@tempa\p@<\p@\scalebox{\@tempa}{\usebox\pandoc@box}%
  \else\usebox{\pandoc@box}%
  \fi%
}
% Set default figure placement to htbp
\def\fps@figure{htbp}
\makeatother
% definitions for citeproc citations
\NewDocumentCommand\citeproctext{}{}
\NewDocumentCommand\citeproc{mm}{%
  \begingroup\def\citeproctext{#2}\cite{#1}\endgroup}
\makeatletter
 % allow citations to break across lines
 \let\@cite@ofmt\@firstofone
 % avoid brackets around text for \cite:
 \def\@biblabel#1{}
 \def\@cite#1#2{{#1\if@tempswa , #2\fi}}
\makeatother
\newlength{\cslhangindent}
\setlength{\cslhangindent}{1.5em}
\newlength{\csllabelwidth}
\setlength{\csllabelwidth}{3em}
\newenvironment{CSLReferences}[2] % #1 hanging-indent, #2 entry-spacing
 {\begin{list}{}{%
  \setlength{\itemindent}{0pt}
  \setlength{\leftmargin}{0pt}
  \setlength{\parsep}{0pt}
  % turn on hanging indent if param 1 is 1
  \ifodd #1
   \setlength{\leftmargin}{\cslhangindent}
   \setlength{\itemindent}{-1\cslhangindent}
  \fi
  % set entry spacing
  \setlength{\itemsep}{#2\baselineskip}}}
 {\end{list}}
\usepackage{calc}
\newcommand{\CSLBlock}[1]{\hfill\break\parbox[t]{\linewidth}{\strut\ignorespaces#1\strut}}
\newcommand{\CSLLeftMargin}[1]{\parbox[t]{\csllabelwidth}{\strut#1\strut}}
\newcommand{\CSLRightInline}[1]{\parbox[t]{\linewidth - \csllabelwidth}{\strut#1\strut}}
\newcommand{\CSLIndent}[1]{\hspace{\cslhangindent}#1}

\makeatletter
\@ifpackageloaded{caption}{}{\usepackage{caption}}
\AtBeginDocument{%
\ifdefined\contentsname
  \renewcommand*\contentsname{Table of contents}
\else
  \newcommand\contentsname{Table of contents}
\fi
\ifdefined\listfigurename
  \renewcommand*\listfigurename{插图索引}
\else
  \newcommand\listfigurename{插图索引}
\fi
\ifdefined\listtablename
  \renewcommand*\listtablename{表索引}
\else
  \newcommand\listtablename{表索引}
\fi
\ifdefined\figurename
  \renewcommand*\figurename{图}
\else
  \newcommand\figurename{图}
\fi
\ifdefined\tablename
  \renewcommand*\tablename{表}
\else
  \newcommand\tablename{表}
\fi
}
\@ifpackageloaded{float}{}{\usepackage{float}}
\floatstyle{ruled}
\@ifundefined{c@chapter}{\newfloat{codelisting}{h}{lop}}{\newfloat{codelisting}{h}{lop}[chapter]}
\floatname{codelisting}{列表}
\newcommand*\listoflistings{\listof{codelisting}{列表索引}}
\usepackage{amsthm}
\theoremstyle{plain}
\newtheorem{proposition}{命题}[section]
\theoremstyle{plain}
\newtheorem{lemma}{引理}[section]
\theoremstyle{plain}
\newtheorem{theorem}{定理}[section]
\theoremstyle{plain}
\newtheorem{corollary}{推论}[section]
\theoremstyle{remark}
\AtBeginDocument{\renewcommand*{\proofname}{证明}}
\newtheorem*{remark}{注记}
\newtheorem*{solution}{解}
\newtheorem{refremark}{注记}[section]
\newtheorem{refsolution}{解}[section]
\makeatother
\makeatletter
\makeatother
\makeatletter
\@ifpackageloaded{caption}{}{\usepackage{caption}}
\@ifpackageloaded{subcaption}{}{\usepackage{subcaption}}
\makeatother

\ifLuaTeX
\usepackage[bidi=basic]{babel}
\else
\usepackage[bidi=default]{babel}
\fi
\babelprovide[main,import]{chinese-hans}
% get rid of language-specific shorthands (see #6817):
\let\LanguageShortHands\languageshorthands
\def\languageshorthands#1{}
\usepackage{bookmark}

\IfFileExists{xurl.sty}{\usepackage{xurl}}{} % add URL line breaks if available
\urlstyle{same} % disable monospaced font for URLs
\hypersetup{
  pdftitle={代数同构视角下的离散 Fourier 变换},
  pdfauthor={钟星宇},
  pdflang={zh-Hans},
  hidelinks,
  pdfcreator={LaTeX via pandoc}}


\title{代数同构视角下的离散 Fourier 变换}
\subtitle{多项式环、求值插值与相似对角化}
\author{钟星宇}
\date{2024-04-20}
\institute{北京理工大学}

\begin{document}
\frame{\titlepage}

\renewcommand*\contentsname{Table of contents}
\begin{frame}[allowframebreaks]
  \frametitle{Table of contents}
  \setcounter{tocdepth}{3}
  \tableofcontents
\end{frame}

\section{1. 从 Fourier 变换到
DFT}\label{ux4ece-fourier-ux53d8ux6362ux5230-dft}

\begin{frame}{Fourier 变换及其卷积性质}
\phantomsection\label{fourier-ux53d8ux6362ux53caux5176ux5377ux79efux6027ux8d28}
\newcommand{\T}{\mathrm{T}}
\newcommand{\Ker}{\operatorname{Ker}}
\newcommand{\Image}{\operatorname{Im}}
\renewcommand{\vec}{\boldsymbol}
\newcommand{\Aut}{\operatorname{Aut}}
\newcommand{\diag}{\operatorname{diag}}
\renewcommand{\i}{\mathrm{i}}
\newcommand{\diff}{\operatorname{d}\!}
\newcommand{\Iso}{\operatorname{Iso}}

\begin{itemize}
\item
  Fourier 变换:将给定函数 \(f\) 映为函数 \(\mathcal F[f]\): \[
  \mathcal F[f](\lambda) := \int_{-\infty}^{\infty} f(t) e^{- \mathrm{i}\lambda t} \operatorname{d}\!t
  \]
\item
  定义函数 \(f\) 和 \(g\) 的卷积 \[
  (f*g)(\lambda) := \int_{-\infty}^{\infty} f(\lambda-x) g(x) \operatorname{d}\!x
  \] 则 Fourier 变换将两个函数的卷积化为逐点乘积,即 \[
  \mathcal F[f*g] = \mathcal F[f] \mathcal F[g]
  \]
\end{itemize}
\end{frame}

\begin{frame}{复数域上的 DFT 及其卷积性质}
\phantomsection\label{ux590dux6570ux57dfux4e0aux7684-dft-ux53caux5176ux5377ux79efux6027ux8d28}
\begin{itemize}
\item
  \emph{离散 Fourier 变换}(Discrete Fourier Transform, DFT):线性空间
  \(\mathbb C^n \to \mathbb C^n\) 上的线性变换 \(F\),将向量
  \(\boldsymbol a  = (a_0,a_1,\dots,a_{n-1})^\mathrm{T}\in \mathbb C^n\)
  映为 \(F \boldsymbol a\),其第 \(i\) 个分量如下所示 \[
  (F \boldsymbol a)_i := \sum_{k=0}^{n-1} \omega_n^{ik} a_i
  \] 这里分量下标从 \(0\)
  开始计数,\(\omega_n := e^{2 \pi \mathrm{i}/ n}\) 是 \(\mathbb C\)
  上的一个 \(n\) 次本原单位根.
\item
  相仿的卷积性:两个向量
  \(\boldsymbol a, \boldsymbol b \in \mathbb C^n\)
  的\emph{循环卷积}定义为 \[
  (\boldsymbol a * \boldsymbol b)_k := \sum_{i + j = k \pmod{n}} a_i b_j
  \] 则 DFT 将两个向量的循环卷积化为\emph{逐项乘积} \(\times\),即 \[
  F(\boldsymbol a * \boldsymbol b) = (F \boldsymbol a) \times (F \boldsymbol b)
  \]
\end{itemize}
\end{frame}

\begin{frame}{矩阵表示}
\phantomsection\label{ux77e9ux9635ux8868ux793a}
在 \(\mathbb C^n\) 的自然基下,变换 \(F\) 有矩阵表示 \[
F = \begin{pmatrix} \omega_n^{ij} \end{pmatrix}_{(i,j)\in n \times n} = \begin{pmatrix}
1 & 1 & \dots & 1 \\
1 & \omega_n & \dots & \omega_n^{n-1} \\
\vdots & \vdots & \ddots & \vdots \\
1 & \omega_n^{n-1} & \dots & \omega_n^{(n-1)(n-1)}
\end{pmatrix}
\]

\begin{itemize}
\tightlist
\item
  卷积性:系数为全体复平面 \(n\) 次单位根的可逆 Vandermonde 矩阵
\item
  正交性:适当单位化后为酉矩阵
\end{itemize}
\end{frame}

\begin{frame}{问题\footnote<.->{{[}1{]}; {[}2{]}; {[}3{]}; {[}4{]};
  {[}5{]}}}
\phantomsection\label{ux95eeux9898}
\begin{itemize}
\item
  DFT 化卷为乘的本质?

  \begin{itemize}
  \tightlist
  \item
    我们给出一大类具备卷积性的线性映射的构造,DFT 将作为特例推出.
  \end{itemize}
\item
  如何从代数角度理解 DFT?

  \begin{itemize}
  \tightlist
  \item
    两个视角:多项式环、矩阵代数
  \item
    两种表现:求值插值、相似对角化
  \item
    一致观点:保加法、保数乘、保乘法的代数同构
  \end{itemize}
\item
  DFT 是否是唯一一类化卷为乘的变换?作为底层结构的 \(\mathbb C\)
  是否可以放宽?

  \begin{itemize}
  \item
    工程上复数乘法运算较慢且具有浮点误差,更换底层代数结构具有实际意义.例如,被称为数论变换(number
    theoretic transforms, NTT)的 DFT 变种就将 \(\mathbb C\)
    替换为有限域 \(\mathbb F_p\) 而同时保留了其卷积性质.
  \item
    我们将其 DFT 扩展至任意整环并证明特定含义下的唯一性.
  \end{itemize}
\end{itemize}
\end{frame}

\section{2. DFT 与多项式环}\label{dft-ux4e0eux591aux9879ux5f0fux73af}

\subsection{\texorpdfstring{2.1 引例:\(\mathbb C[x]\)、求值插值与复数域
DFT}{2.1 引例:\textbackslash mathbb C{[}x{]}、求值插值与复数域 DFT}}\label{ux5f15ux4f8bmathbb-cxux6c42ux503cux63d2ux503cux4e0eux590dux6570ux57df-dft}

\begin{frame}{\(\mathbb C[x]\) 与循环卷积}
\phantomsection\label{mathbb-cx-ux4e0eux5faaux73afux5377ux79ef}
设不超过 \(n-1\) 次的多项式
\(f(x) = \sum_{k=0}^{n-1} a_k x^k\),\(g(x) = \sum_{k=0}^{n-1} b_k x^k\).二者的多项式乘积由
\emph{Cauchy 乘积}给出 \[
f(x) g(x) = \sum_{i=0}^{n-1} a_i x^i \sum_{j=0}^{n-1} b_j x^j = \sum_{k=0}^{2n-2} x^k \sum_{i+j = k} a_i b_j
\] 令
\(\boldsymbol a := (a_0,a_1,\dots,a_{n-1})^\mathrm{T}\),\(\boldsymbol b := (b_0,b_1,\dots,b_{n-1})^\mathrm{T}\),回顾循环卷积定义
\[
(\boldsymbol a * \boldsymbol b)_k := \sum_{i + j = k \pmod{n}} a_i b_j
\] 可见 Cauchy 乘积与循环卷积尚有区别.稍加改动,若在模 \(x^n - 1\)
的意义下------即商环 \(\mathbb C[x]/(x^n-1)\) 中计算,则二者相合: \[
f(x) g(x) = \sum_{k=0}^{n-1} x^k \sum_{i + j = k \pmod{n}} a_i b_j \pmod{x^n - 1}
\]
\end{frame}

\begin{frame}{\(\mathbb C[x]\) 与复数域 DFT}
\phantomsection\label{mathbb-cx-ux4e0eux590dux6570ux57df-dft}
DFT 亦有在 \(\mathbb C[x]\) 上的表示.给定
\(\boldsymbol a := (a_0,a_1,\dots,a_{n-1})^\mathrm{T}\in \mathbb C^n\),其对应多项式
\(f(x) = \sum_{k=0}^{n-1} a_k x^k\) 次数不超过 \(n-1\) 次,则 \[
(F \boldsymbol a)_i = \sum_{k=0}^{n-1} \omega_n^{ik} \boldsymbol a_i = \sum_{k=0}^{n-1} \boldsymbol a_i (\omega_n^i)^k = f(\omega_n^i)
\] 恰为 \(f(x)\) 分别在 \(n\) 个 \(\mathbb C\) 上 \(n\)
次单位根处\emph{多点求值}的结果.

\begin{itemize}
\tightlist
\item
  可逆性:\(n\) 点唯一确定一个不超过 \(n-1\) 次的多项式(\emph{Lagrange
  插值})
\item
  线性性:\((af+bg) (\omega_n^i) = a f(\omega_n^i) + b g(\omega_n^i)\)
\item
  卷积性:将取模乘法化为点值逐项相乘,再次与 \(\mathbb C^n\)
  上的表现相合 \[
  \begin{aligned}
  F(\boldsymbol a * \boldsymbol b) &= (F \boldsymbol a) \times (F \boldsymbol b) \\
  (fg)(\omega_n^i) &= f(\omega_n^i) g(\omega_n^i)
  \end{aligned}
  \]
\end{itemize}
\end{frame}

\begin{frame}{小结}
\phantomsection\label{ux5c0fux7ed3}
\begin{itemize}
\item
  \(\mathbb C^n\) 与 \(\mathbb C[x]\) 视角下的 DFT:

  \begin{itemize}
  \item
    \(\mathbb C^n\):作为以单位根为参数的 Vandermonde 矩阵,DFT 是
    \(\mathbb C^n\) 上的可逆线性变换,将向量间的循环卷积 \(*\)
    化为逐项乘积 \(\times\).
  \item
    \(\mathbb C[x]\):作为单位根处的多点求值插值,DFT 在全体不超过
    \(n-1\) 次的多项式和 \(\mathbb C^n\)
    间建立起线性同构关系,将多项式乘积化为函数值逐点乘积.
  \end{itemize}
\item
  化卷为乘,就是把多项式环上的取模乘法变为 \(\mathbb C^n\)
  上的逐项乘积,DFT 保持了两个代数结构间的乘法.

  \begin{itemize}
  \tightlist
  \item
    \(\mathbb C[x]\) 作为环结构乘法自然,在多项式环上刻画 DFT 较在
    \(\mathbb C^n\) 上强行定义循环卷积具有优越性.
  \end{itemize}
\end{itemize}
\end{frame}

\subsection{2.2
整环上的推广}\label{ux6574ux73afux4e0aux7684ux63a8ux5e7f}

\begin{frame}{代数、代数同构与直积}
\phantomsection\label{ux4ee3ux6570ux4ee3ux6570ux540cux6784ux4e0eux76f4ux79ef}
\begin{itemize}
\item
  整环:无零因子交换幺环
\item
  设 \(R\) 是一整环,若 \((A,+,\times)\) 为一环且配备了与乘法 \(\times\)
  相容的 \(R\)-数乘 \(\cdot\),则称 \(A\) 是一
  \(R\)-代数,不至混淆时简称代数.

  \begin{itemize}
  \tightlist
  \item
    整环 \(R\) 自身也可视为一个代数.
  \end{itemize}
\item
  我们将 \(R^n\) 理解为作为代数的 \(R\) 的直积,即
  \(R^n = R \times R \times \dots \times R\).直积的加法、数乘和乘法均在逐项意义下定义.
\item
  保持代数间加法、数乘和乘法的双射被称为代数同构.
\end{itemize}
\end{frame}

\begin{frame}{几个观察与整环的优势}
\phantomsection\label{ux51e0ux4e2aux89c2ux5bdfux4e0eux6574ux73afux7684ux4f18ux52bf}
\begin{itemize}
\item
  关于引例的若干观察:

  \begin{itemize}
  \item
    DFT 是 \(\mathbb C[x] / (x^n-1) \to R^n\)
    的一个代数同构,具体做法是在单位根处多点求值插值
  \item
    求值插值在任意 \(n\) 个不同位置进行即可,单位根不是本质要求
  \item
    商环 \(\mathbb C[x] / (x^n-1)\)
    带来了与循环卷积对应的多项式取模乘法,还蕴含着``不超过 \(n-1\)
    次''为求值插值带来的单与满
  \item
    \emph{第一同构定理}:设 \(f: R \to S\) 是环同态,则 \(f\)
    诱导出环同构 \(R / \operatorname{Ker}f \cong \operatorname{Im}f\)
  \end{itemize}
\item
  选取整环作为底层代数结构的理由:

  \begin{itemize}
  \tightlist
  \item
    交换:确保求值操作是同态
  \item
    保留环上整除的结构和多项式根与因子的关系(带余除法、余式定理)
  \item
    在唯一性证明中发挥作用
  \end{itemize}
\end{itemize}
\end{frame}

\begin{frame}[fragile]{商环到直积的代数同构}
\phantomsection\label{ux5546ux73afux5230ux76f4ux79efux7684ux4ee3ux6570ux540cux6784}
下面固定 \(R\) 是一整环.令 \(C\) 是 \(R\)
的一有限子集,由若干一次多项式乘积 \(\prod_{c \in C} (x-c)\) 生成的
\(R[x]\) 上的理想记为 \(\left( \prod_{c \in C} (x-c) \right)\).

用记号 \(R^C\) 代表全体 \(C\) 上的 \(R\) 值函数构成的集合.\(R^C\)
与其上定义的函数逐点加法、数乘和乘法构成一个代数,自然也与 \(R^n\)
代数同构.

\begin{theorem}[]\protect\hypertarget{thm-dft-exist}{}\label{thm-dft-exist}

多项式商环 \(R[x] / \left( \prod_{c \in C} (x-c) \right)\) 与代数直积
\(R^C\) 代数同构.

\end{theorem}

\begin{figure}

\centering{

\begin{Shaded}
\begin{Highlighting}[]
\NormalTok{\textbackslash{}begin\{tikzcd\}}
\NormalTok{\{R[x]\} \textbackslash{}arrow[rr, "\textbackslash{}varphi"] \textbackslash{}arrow[d]                                       \&  \& R\^{}C \textbackslash{}\textbackslash{}}
\NormalTok{\{R[x] / \textbackslash{}left( \textbackslash{}prod\_\{c \textbackslash{}in C\} (x{-}c) \textbackslash{}right)\} \textbackslash{}arrow[rru, "\textbackslash{}bar \textbackslash{}varphi", dashed] \&  \&    }
\NormalTok{\textbackslash{}end\{tikzcd\}}
\end{Highlighting}
\end{Shaded}

}

\caption{\label{fig-dft-exist}定理~\ref{thm-dft-exist} 构造示意图}

\end{figure}%
\end{frame}

\begin{frame}{构造}
\phantomsection\label{ux6784ux9020}
考察 \(R[x]\) 到 \(R^C\) 上的代数同态
\(\varphi: f \mapsto (C \ni x \mapsto f(x))\),其含义为在每一
\(c \in C\) 处对多项式 \(f\) 进行求值.

\begin{itemize}
\item
  \(\varphi\) 的核:

  \[
  \operatorname{Ker}\varphi = \{f \in R[x]: f(C)=\{0\}\} = \left( \prod_{c \in C} (x-c) \right)
  \]
\item
  \(\varphi\) 的像:对每个 \(c \in C\) 对应的理想 \((x-c)\)
  应用中国剩余定理就有 \(\operatorname{Im}\varphi = R^C\).
\end{itemize}

故由第一同构定理,\(\varphi\) 诱导的 \[
\bar \varphi: R[x] / \left( \prod_{c \in C} (x-c) \right) \to R^C
\] 是一同构映射.
\end{frame}

\begin{frame}{DFT:代数同构的特例}
\phantomsection\label{dftux4ee3ux6570ux540cux6784ux7684ux7279ux4f8b}
作为上一定理的特例,DFT 在单位根处求值插值.若 \(\omega_n\) 为内嵌于
\(R\) 的某一 \(n\) 阶循环(乘法)群的生成元,则称其为 \emph{\(R\) 上的
\(n\) 次本原单位根}.

\begin{corollary}[]\protect\hypertarget{cor-dft-exist}{}\label{cor-dft-exist}

若 \(R\) 上存在 \(n\) 次本原单位根 \(\omega_n\),则多项式 \[
x^n - 1 = \prod_{k=0}^{n-1} (x - \omega_n^k)
\] 故 \(R[x] / \left( x^n - 1 \right)\) 与 \(R^n\)
代数同构.我们便称二者间的代数同构为 \emph{\(R\) 上的 \(n\) 点 DFT}.

\end{corollary}
\end{frame}

\subsection{2.3
唯一性的讨论}\label{ux552fux4e00ux6027ux7684ux8ba8ux8bba}

\begin{frame}[fragile]{全体代数同构的结构}
\phantomsection\label{ux5168ux4f53ux4ee3ux6570ux540cux6784ux7684ux7ed3ux6784}
\begin{figure}

\centering{

\begin{Shaded}
\begin{Highlighting}[]
\NormalTok{\textbackslash{}begin\{tikzcd\}}
\NormalTok{\{R[x]/(m(x))\} \textbackslash{}arrow[rr, "\textbackslash{}bar \textbackslash{}varphi"] \&  \& R\^{}n \textbackslash{}arrow["?"\textquotesingle{}, dashed, loop, distance=2em, in=65, out=355]}
\NormalTok{\textbackslash{}end\{tikzcd\}}
\end{Highlighting}
\end{Shaded}

}

\caption{\label{fig-big-picture-1}}

\end{figure}%

已经建立 \(R[x]/(m(x)) \to R^n\) 的同构关系,这里 \(m(x)\)
是若干一次因式的乘积.但这种同构或不止一种.为研究其是否在某种意义下具有唯一性,需研究全体同构
\(\operatorname{Iso}(R[x]/(m(x)),R^n)\) 的结构.该问题化归为研究 \(R^n\)
上全体自同构 \(\operatorname{Aut}(R^n)\) 的结构.

\begin{proposition}[]\protect\hypertarget{prp-dft-unique}{}\label{prp-dft-unique}

设 \(\mathcal A\) 是一与 \(R^n\) 同构的任一代数.固定代数同构
\(\varphi: \mathcal A \to R^n\),则任一 \(\mathcal A \to R^n\)
的代数同构 \(f\) 都具有形式 \(f = p \varphi\),这里
\(p \in \operatorname{Aut}(R^n)\).

\end{proposition}
\end{frame}

\begin{frame}{\(R^n\) 上的自同构}
\phantomsection\label{rn-ux4e0aux7684ux81eaux540cux6784}
设 \(\boldsymbol e_1,\dots,\boldsymbol e_n\) 是 \(R^n\) 上的自然基,设
\(\sigma \in S_n\) 是有限集 \(\{0,1,\dots,n-1 \}\) 上的一个置换.定义
\(R^n\) 上由置换 \(\sigma\) 诱导的模自同构 \[
P_\sigma: \boldsymbol e_k \mapsto \boldsymbol e_{\sigma(k)}
\] 容易验证 \(P_\sigma\) 保持逐项乘法,因此它也是 \(R^n\)
上的代数自同构.

下面的引理刻画了 \(R^n\) 上代数自同构的形式.

\begin{lemma}[]\protect\hypertarget{lem-perm}{}\label{lem-perm}

全体 \(P_\sigma\) 构成 \(R^n\) 上全体代数自同构,即 \[
\operatorname{Aut}(R^n) = \{ P_\sigma : \sigma \in S_n \}
\]

\end{lemma}

整环、可逆性、保乘法、保线性的综合应用使得 \(P_\sigma\)
的矩阵表示每行每列有且仅有一个 \(1\).
\end{frame}

\begin{frame}{DFT 的唯一性}
\phantomsection\label{dft-ux7684ux552fux4e00ux6027}
\begin{corollary}[]\protect\hypertarget{cor-dft-unique}{}\label{cor-dft-unique}

设 \(f\) 是任一 \(R\) 上的 \(n\) 点 DFT,则任何 \(R\) 上的 \(n\) 点 DFT
\(g\) 都具有形式 \(g = P_\sigma f\),这里 \(f\) 是一事先固定的 \(n\) 点
DFT.

\end{corollary}

作为推论,\(n\) 点 DFT 共有 \(n!\)
种.这一结果的显著性在于,只要不计求值得到的 \(n\) 个点值在 \(R^n\)
上的排列顺序,DFT 是唯一满足卷积性质的可逆线性映射.
\end{frame}

\section{3. DFT 与矩阵代数}\label{dft-ux4e0eux77e9ux9635ux4ee3ux6570}

\begin{frame}[fragile]{第二个视角:矩阵代数}
\phantomsection\label{ux7b2cux4e8cux4e2aux89c6ux89d2ux77e9ux9635ux4ee3ux6570}
我们建立了

\begin{figure}

\centering{

\begin{Shaded}
\begin{Highlighting}[]
\NormalTok{\textbackslash{}begin\{tikzcd\}}
\NormalTok{\{R[x]/(m(x))\} \textbackslash{}arrow[rr, "\textbackslash{}bar \textbackslash{}varphi"] \&  \& R\^{}n \textbackslash{}arrow["P\_\textbackslash{}sigma"\textquotesingle{}, loop, distance=2em, in=65, out=355]}
\NormalTok{\textbackslash{}end\{tikzcd\}}
\end{Highlighting}
\end{Shaded}

}

\caption{\label{fig-big-picture-2}}

\end{figure}%

这一交换图可以继续扩展.将视角从多项式环转向矩阵代数,我们将看到,DFT
不仅是多项式环上的求值插值,更体现为矩阵代数上的相似对角化.

简单起见,下面只考察代数闭域的情况,并用域的常用记号 \(K\) 替代 \(R\).
\end{frame}

\begin{frame}[fragile]{相似对角化}
\phantomsection\label{ux76f8ux4f3cux5bf9ux89d2ux5316}
设 \(C\) 是域 \(K\) 上的 \(n\)
阶可对角化矩阵,特征值两两不同.设其特征多项式(或最小多项式)为
\(m(x)\),\(\Lambda\) 为其对角化得到的矩阵.\(K[C]\) 和 \(K[\Lambda]\)
分别是矩阵 \(C\) 和 \(\Lambda\) 在 \(K^{n \times n}\) 上生成的代数.

\begin{figure}

\centering{

\begin{Shaded}
\begin{Highlighting}[]
\NormalTok{\textbackslash{}begin\{tikzcd\}}
\NormalTok{\{K[x]/(m(x))\} \textbackslash{}arrow[rr, "\textbackslash{}bar \textbackslash{}varphi"] \textbackslash{}arrow[d, "?", dashed] \&  \& K\^{}n \textbackslash{}arrow[d, "\textbackslash{}operatorname\{diag\}"] \textbackslash{}arrow["P\_\textbackslash{}sigma"\textquotesingle{}, loop, distance=2em, in=65, out=355] \textbackslash{}\textbackslash{}}
\NormalTok{\{K[C]\} \textbackslash{}arrow[rr, "\textbackslash{}text\{diagonalization\}"]                                     \&  \& \{K[\textbackslash{}Lambda]\}}
\NormalTok{\textbackslash{}end\{tikzcd\}}
\end{Highlighting}
\end{Shaded}

}

\caption{\label{fig-big-picture-3}}

\end{figure}%

能够对角化 \(C\) 的矩阵也同时对角化了 \(K[C]\)
中的任意矩阵.若设这一对角化矩阵为 \(F\),则 \(A \mapsto F^{-1} A F\)
便规定了一个 \(K[C] \to K[\Lambda]\) 的代数同构.\(K^n\) 与
\(K[\Lambda]\) 的代数同构是平凡的.下面来建立 \(K[x] / m(x)\) 与
\(K[C]\) 间的联系.
\end{frame}

\begin{frame}[fragile]{}
\phantomsection\label{section}
\begin{theorem}[]\protect\hypertarget{thm-minpoly-iso}{}\label{thm-minpoly-iso}

\(K[x] / m(x)\) 与 \(K[C]\) 代数同构.

\end{theorem}

仍然考察 \(K[x] \to K[C]\) 自然的``代入''
\(\psi : f \mapsto f(C)\).\(C\) 的全体零化多项式恰为 \(m(x)\) 生成的
\(K[x]\) 上的理想,因此 \(\operatorname{Ker}\psi = (m(x))\).\(\psi\)
的满射性平凡.用第一同构定理就得到结论.

\begin{figure}

\centering{

\begin{Shaded}
\begin{Highlighting}[]
\NormalTok{\textbackslash{}begin\{tikzcd\}}
\NormalTok{\{K[x]\} \textbackslash{}arrow[rr, "\textbackslash{}psi"] \textbackslash{}arrow[d]                                       \&  \& K[C] \textbackslash{}\textbackslash{}}
\NormalTok{\{K[x] / \textbackslash{}left( m(x) \textbackslash{}right)\} \textbackslash{}arrow[rru, "\textbackslash{}bar \textbackslash{}psi", dashed] \&  \&    }
\NormalTok{\textbackslash{}end\{tikzcd\}}
\end{Highlighting}
\end{Shaded}

}

\caption{\label{fig-dft-exist}定理~\ref{thm-minpoly-iso} 证明示意图}

\end{figure}%
\end{frame}

\begin{frame}{对角化矩阵的显式构造}
\phantomsection\label{ux5bf9ux89d2ux5316ux77e9ux9635ux7684ux663eux5f0fux6784ux9020}
我们取一类性质更好的可对角化矩阵 \(C\) 来显式构造出用于对角化 \(K[C]\)
的矩阵.这一矩阵定义为 \[
C = 
\begin{pmatrix}
0 & 1 & 0 &\ldots & 0\\
0 & 0 & 1 &\ldots & 0 \\
0 & 0 & 0 &\ldots & 0 \\
\vdots & \vdots & \vdots & \ddots &1\\
-c_0 & -c_1 & -c_2 & \ldots & -c_{n-1}
\end{pmatrix}
\] 它被称为多项式 \(m(x) = c^n + a_{n-1} c^{n-1} + \dots + c_0\)
的\emph{友矩阵}(companion matrix).

\begin{itemize}
\item
  直接计算,\(C\) 的特征多项式和最小多项式恰为 \(m(x)\).
\item
  直接验证,特征值 \(\lambda_k\) 对应特征向量为
  \((1,\lambda_k,\dots,\lambda_k^{n-1})^\mathrm{T}\).
\end{itemize}
\end{frame}

\begin{frame}{}
\phantomsection\label{section-1}
可见 Vandermonde 矩阵 \[
F = \begin{pmatrix}
1 & 1 & \dots & 1 \\
\lambda_0 & \lambda_1 & \dots & \lambda_{n-1} \\
\vdots & \vdots & \ddots & \vdots \\
\lambda_0^{n-1} & \lambda_1^{n-1} & \dots & \lambda_{n-1}^{n-1}
\end{pmatrix}
\] 正是将友矩阵 \(C\) 对角化的矩阵. \[
F^{-1} C F = \Lambda = \operatorname{diag}(\lambda_0,\lambda_1,\dots,\lambda_{n-1})
\] 注意到 \(K[C]\) 也被对角化 \(C\) 的矩阵同时对角化,故
\(A \mapsto F^{-1} A F\) 确为 \(K[C] \to K[\Lambda]\)
的代数同构,与先前的关于对角化的讨论结果一致.
\end{frame}

\begin{frame}{循环矩阵的对角化}
\phantomsection\label{ux5faaux73afux77e9ux9635ux7684ux5bf9ux89d2ux5316}
特别地,若取 \[
C = 
\begin{pmatrix}
0 & 1 & 0 &\ldots & 0\\
0 & 0 & 1 &\ldots & 0 \\
0 & 0 & 0 &\ldots & 0 \\
\vdots & \vdots & \vdots & \ddots &1\\
1 & 0 & 0 & \ldots & 0
\end{pmatrix}
\] 它是基本循环矩阵,对应最小多项式 \(m(x) = x^n - 1\).\(C\) 生成的代数
\(K[C]\) 即 \(K^{n \times n}\) 上的全体循环矩阵.此时 DFT 体现为利用 DFT
矩阵 \[
F = \begin{pmatrix}
1 & 1 & \dots & 1 \\
1 & \omega_n & \dots & \omega_n^{n-1} \\
\vdots & \vdots & \ddots & \vdots \\
1 & \omega_n^{n-1} & \dots & \omega_n^{(n-1)(n-1)}
\end{pmatrix}
\] 对循环矩阵进行对角化的过程.
\end{frame}

\begin{frame}[fragile]{结语}
\phantomsection\label{ux7ed3ux8bed}
以刻画 DFT 的卷积性质为目标,以代数同构为构造手段,我们为理解 DFT
的代数含义提供了两个视角:

\begin{itemize}
\tightlist
\item
  DFT 是多项式商环上的多点求值插值
\item
  DFT 是矩阵代数上的相似对角化
\end{itemize}

可见 DFT
背后的代数理论非常丰富,不失为联系起本科阶段代数课程的有趣实例,亦体现出代数工具与视角在工程实践中的强大效用.

\begin{figure}

\centering{

\begin{Shaded}
\begin{Highlighting}[]
\NormalTok{\textbackslash{}begin\{tikzcd\}}
\NormalTok{\{K[x]/(m(x))\} \textbackslash{}arrow[rr, "\textbackslash{}bar \textbackslash{}varphi"] \textbackslash{}arrow[d, "\textbackslash{}bar \textbackslash{}psi"] \&  \& K\^{}n \textbackslash{}arrow[d, "\textbackslash{}operatorname\{diag\}"] \textbackslash{}arrow["P\_\textbackslash{}sigma"\textquotesingle{}, loop, distance=2em, in=65, out=355] \textbackslash{}\textbackslash{}}
\NormalTok{\{K[C]\} \textbackslash{}arrow[rr, "\textbackslash{}text\{diagonalization\}"]                                     \&  \& \{K[\textbackslash{}Lambda]\}}
\NormalTok{\textbackslash{}end\{tikzcd\}}
\end{Highlighting}
\end{Shaded}

}

\caption{\label{fig-big-picture-final}}

\end{figure}%
\end{frame}

\begin{frame}{Acknowledgements}
\phantomsection\label{acknowledgements}
The speaker wishs to express his gratitude to

\begin{itemize}
\item
  Professor Feng Zhang, School of Information and Electronics, BIT, for
  his long-term guidance on this subject.
\item
  Professor Peng Cao, School of Mathematics and Statistics, BIT, for his
  valuable advice on the presentation.
\end{itemize}
\end{frame}

\begin{frame}{}
\phantomsection\label{section-2}
Thanks for listening!

\phantomsection\label{refs}
\begin{CSLReferences}{0}{0}
\bibitem[\citeproctext]{ref-agarwal_number_1975}
\CSLLeftMargin{{[}1{]} }%
\CSLRightInline{R. C. Agarwal 和 C. S. Burrus, {《Number theoretic
transforms to implement fast digital convolution》}, \emph{Proceedings
of the IEEE}, 卷 63, 期 4, 页 550--560, 4月 1975, doi:
\href{https://doi.org/10.1109/PROC.1975.9791}{10.1109/PROC.1975.9791}.}

\bibitem[\citeproctext]{ref-nicholson_algebraic_1971}
\CSLLeftMargin{{[}2{]} }%
\CSLRightInline{P. J. Nicholson, {《Algebraic theory of finite fourier
transforms》}, \emph{Journal of Computer and System Sciences}, 卷 5, 期
5, 页 524--547, 10月 1971, doi:
\href{https://doi.org/10.1016/S0022-0000(71)80014-4}{10.1016/S0022-0000(71)80014-4}.}

\bibitem[\citeproctext]{ref-furer_faster_2009}
\CSLLeftMargin{{[}3{]} }%
\CSLRightInline{M. Fürer, {《Faster {Integer} {Multiplication}》},
\emph{SIAM Journal on Computing}, 卷 39, 期 3, 页 979--1005, 1月 2009,
doi: \href{https://doi.org/10.1137/070711761}{10.1137/070711761}.}

\bibitem[\citeproctext]{ref-amiot_music_2016}
\CSLLeftMargin{{[}4{]} }%
\CSLRightInline{E. Amiot, \emph{Music {Through} {Fourier} {Space}}. 收入
Computational {Music} {Science}. Cham: Springer International
Publishing, 2016. doi:
\href{https://doi.org/10.1007/978-3-319-45581-5}{10.1007/978-3-319-45581-5}.}

\bibitem[\citeproctext]{ref-baraquin_uniqueness_2023}
\CSLLeftMargin{{[}5{]} }%
\CSLRightInline{I. Baraquin 和 N. Ratier, {《Uniqueness of the discrete
{Fourier} transform》}, \emph{Signal Processing}, 卷 209, 页 109041, 8月
2023, doi:
\href{https://doi.org/10.1016/j.sigpro.2023.109041}{10.1016/j.sigpro.2023.109041}.}

\end{CSLReferences}
\end{frame}




\end{document}
