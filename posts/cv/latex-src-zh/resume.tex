\documentclass{resume}
\usepackage{linespacing_fix}
\usepackage{cite}
\usepackage{hyperref}
\usepackage{parcolumns}
\hypersetup{
    colorlinks=true,
    linkcolor=cyan,
    filecolor=magenta,      
    urlcolor=blue,
}

\begin{document}
\pagenumbering{gobble}

%***"%"后面的所有内容是注释而非代码,不会输出到最后的PDF中
%***使用本模板,只需要参照输出的PDF,在本文档的相应位置做简单替换即可
%***修改之后,输出更新后的PDF,只需要点击Overleaf中的“Recompile”按钮即可

%在大括号内填写其他信息,最多填写4个,但是如果选择不填信息,
%那么大括号必须空着不写,而不能删除大括号。
%\otherInfo后面的四个大括号里的所有信息都会在一行输出
%如果想要写两行,那就用两次这个指令(\otherInfo{}{}{}{})即可


%***********个人信息**************
\MyName{钟星宇}
\sepspace
\SimpleEntry{\textbf{学校}\quad 北京理工大学}
\SimpleEntry{\textbf{专业}\quad 数学与应用数学(强基班)}
\SimpleEntry{\textbf{邮箱}\quad sun123zxy@qq.com}
\SimpleEntry{\textbf{电话}\quad +86 17723294568}
\SimpleEntry{\textbf{主页}\quad \href{https://scholar.sun123zxy.top/}{https://scholar.sun123zxy.top}}

%************照片**************
%照片需要放到images文件夹下,名字必须是you.jpg,注意.jpg后缀(可以去resume.cls第101行处修改),如果不需要照片可以不添加此行命令
%0.15的意思是,照片的宽度是页面宽度的0.15倍,调整大小,避免遮挡文字
\yourphoto{0.1}

%***********教育背景**************
\section{教育背景}
%***第一个大括号里的内容向左对齐,第二个大括号里的内容向右对齐
%***\textbf{}括号里的字是粗体,\textit{}括号里的字是斜体
\datedsubsection{\textbf{北京理工大学} \quad 数学与应用数学(强基班) \quad \textit{本科}}{2022.09 -- }
\begin{itemize}
  \item \textbf{国家奖学金}一项\quad\textbf{三星奖学金}一项\quad\textbf{校一等奖学金}五项\quad\textbf{校级优秀学生标兵}一项\quad\textbf{校级优秀学生}一项
  \item \textbf{平均学分绩} 93.6 \quad\textbf{GPA} 3.9 / 4 \quad\textbf{成绩排名} 1 / 27 \quad\textbf{综测排名} 1 / 27 \quad \textbf{优良率} 100\% \quad(截至第五学期)
  \item \textbf{专业课程}\quad 数学分析、高等代数、解析几何、概率论、复变函数、实变函数、近世代数、泛函分析、点集拓扑、微分流形、常微分方程、偏微分方程、数理统计、数值计算方法、离散数学、初等数论、组合学、代数几何初步、并行计算、李代数量子群范畴化、C 语言程序设计基础等
\end{itemize}

\section{学术经历}

\datedsubsection{\textbf{学术报告} \quad 北理工数学拔尖、强基学子赴中科大交流活动 \quad \textit{报告人}}{2024.04.20}
\textbf{指导老师}\quad 曹鹏 \quad 副教授 \quad 北京理工大学 \quad 数学与统计学院
\begin{itemize}
    \item 赴中国科学技术大学交流学习并作学术报告《代数同构视角下的离散 Fourier 变换》.
\end{itemize}

\datedsubsection{\textbf{大学生创新创业项目} \quad Fourier 矩阵唯一性 \quad \textit{第一负责人}}{2022.12 -- 2023.12}
\textbf{指导老师}\quad 张峰 \quad 副教授 \quad 北京理工大学 \quad 信息与电子学院
\begin{itemize}
    \item 作为第一负责人完成市、校级大创《将循环卷积转化为乘积的矩阵是否只有傅里叶矩阵?》,探讨具有电子信息应用背景的数学理论问题.
\end{itemize}

\datedsubsection{\textbf{国际项目} \quad 2025 FRP Winter Programme in Mordern Advanced Deep Learning}{2025.02}
\textbf{Advisor}\quad Prof. Jose Hernandez-Lobato \quad
University of Cambridge
\begin{itemize}
    \item A two-week winter program held in University of Cambridge on modern deep learning theories. Group presentations on hands-on projects were required at the end of the program. Full attendance and received the special award (3 out of 15).
\end{itemize}

\datedsubsection{\textbf{讨论班} \quad 矩阵联合上三角化 \quad \textit{参与}}{2023.11 -- 2024.06}
\textbf{指导老师}\quad 曹鹏 \quad 副教授 \quad 北京理工大学 \quad 数学与统计学院
\begin{itemize}
    \item 参与讨论班.参考教材 H. Radjavi and P. Rosenthal, Simultaneous Triangularization,讨论了其前两章内容.
\end{itemize}

\section{竞赛获奖}
\subsection{\textbf{数学竞赛}}

\begin{itemize}
\item\dated{2024 年阿里巴巴全球数学竞赛 决赛入围}{2024.04.15}
\item\dated{2022 年北京理工大学“明理杯”新生数学竞赛 一等奖}{2022.10}
\end{itemize}

\begin{parcolumns}{2}
\colchunk{
\textbf{全国大学生数学竞赛}
\begin{itemize}
\item \dated{北京赛区预赛数学专业组 A 类三等奖}{2023.11.11}
\item \dated{北京赛区预赛数学专业组 A 类三等奖}{2024.11.09}
\end{itemize}
}
\colchunk{
\textbf{北京市大学生数学竞赛}
\begin{itemize}
\item \dated{数学专业组 A 类二等奖}{2023.11.11}
\item \dated{数学专业组 A 类三等奖}{2024.11.09}
\end{itemize}
}
\end{parcolumns}

\subsection{\textbf{算法竞赛}}

\begin{parcolumns}{2}
\colchunk{
\textbf{ICPC 国际大学生程序设计竞赛}
\begin{itemize}
\item \dated{2023 年亚洲区域赛(西安)\textcolor{brown}{\textbf{铜奖}}}{2023.10.22}
\item \dated{2023 年亚洲区域赛(合肥)\textcolor{Gray}{\textbf{银奖}}}{2023.11.25}
\item \dated{2024 年亚洲区域赛(南京)\textcolor{brown}{\textbf{铜奖}}}{2024.11.03}
\end{itemize}
}
\colchunk{
\textbf{CCPC 中国大学生程序设计竞赛}
\begin{itemize}
\item \dated{2023 年全国邀请赛(湖南)\textcolor{Gray}{\textbf{银奖}}}{2023.05.28}
\item \dated{2023 年国赛(哈尔滨站)\textcolor{Gray}{\textbf{银奖}}}{2023.11.05}
\item \dated{2024 年全国邀请赛(广东)\textcolor{Goldenrod}{\textbf{金奖}}}{2024.05.26}
\item \dated{2024 年国赛(郑州站)\textcolor{Gray}{\textbf{银奖}}}{2024.11.17}
\end{itemize}
}
\end{parcolumns}

\begin{itemize}
    \item \dated{蓝桥杯、百度之星、程序设计校赛等其它国、省、校级奖项若干}{2022--2024}
\end{itemize}

\subsection{\textbf{其它竞赛}}

\begin{itemize}
\item \dated{2023 年全国大学生统计建模大赛北京赛区选拔赛本科生组二等奖}{2023.08}
\end{itemize}

\section{社会工作}

\datedsubsection{\textbf{课程讲授} \quad 2023 北理工 / 延安大学 ACM 暑期集训 \quad \textit{课程主讲人}}{2023.07 -- 2023.08}
\begin{itemize}
    \item 参与 2023 年北京理工大学计算机学院暑期实践重点团“‘科技强国’—暑期ACM集训”实践工作,作为算法课程主讲团队成员,为北京理工大学和延安大学学子讲授快速 Fourier 变换、数论变换相关算法与数学原理.
\end{itemize}


\datedsubsection{\textbf{试题命制} \quad 多场国、校级算法竞赛 \quad \textit{命题团队成员}}{2023 -- 2024}
\begin{itemize}
    \item 参与 2023 年北京理工大学程序设计校赛、2024 年 ICPC 国际大学生程序设计竞赛亚洲区域赛(昆明)的试题命制工作.
\end{itemize}

\datedsubsection{\textbf{研讨主持} \quad 多场数学研讨活动 / 讨论班 \quad \textit{联合主持人}}{2024 -- 2025}
\begin{itemize}
    \item 联合主持 2024 年春北京理工大学 2022 级数学强基班实变函数、复变函数课后研讨活动.联合主持 2024 年秋抽象代数和近世代数讨论班.
\end{itemize}

\section{其它技能}
\paragraph{}{\textbf{计算机}\quad 熟练 Python / C++ / Web 等多项编程语言,熟悉 Linux 操作系统,熟练 Git 等团队协作开发工具.有较复杂静态网站编写和搭建经验.有小体量命令行工具开发经验.}
\paragraph{}{\textbf{排版工具}\quad 熟练使用 LaTeX 排版工具.熟练使用 Pandoc / Quarto 等高阶出版系统.编写过适用于 Quarto 的 Lua filter 扩展.}
\paragraph{}{\textbf{算法}\quad 前 OI / XCPC 算法竞赛选手,熟练掌握常见算法和数据结构. \quad \textbf{Codeforces} 1951}
\paragraph{}{\textbf{数学形式化}\quad 了解和正在学习基于 Lean 语言的数学形式化工作.}
\paragraph{}{\textbf{语言}\quad \textbf{IELTS Academic} 7.5 (8.5 / 8.5 / 6.5 / 6) \quad \textbf{CET-4} 650 \quad \textbf{CET-6} 581 \quad \textbf{CET-SET4} A}

\sepspace
\end{document}